\documentclass[12pt]{article}
\usepackage{amsmath}
\usepackage{amssymb}
\usepackage{geometry}
\geometry{margin=1in}

\title{CG-CEL (CEL2new) Calculations}
\date{}

\begin{document}

\maketitle

\section*{Algorithm}

CG-CEL (Contested Garment - Constrained Equal Losses) has two cases:

\textbf{Case 1:} If $E \leq \frac{1}{2}\sum c_i$, use CEA on half-claims:
\begin{equation*}
\text{Allocation}_i = \min\left(\frac{c_i}{2}, \lambda\right), \quad \text{where } \sum_{i=1}^{n} \min\left(\frac{c_i}{2}, \lambda\right) = E
\end{equation*}

\textbf{Case 2:} If $E > \frac{1}{2}\sum c_i$, give everyone half-claim first, then distribute remainder using CEL:
\begin{equation*}
\text{Allocation}_i = \frac{c_i}{2} + \max\left(0, \frac{c_i}{2} - \mu\right)
\end{equation*}
where $\mu$ satisfies $\sum \max(0, \frac{c_i}{2} - \mu) = E - \frac{1}{2}\sum c_i$

\section*{E = 20}

\textbf{Case 1:} $c = (25, 25, 25, 25)$, $E = 20$
\begin{align*}
h &= (12.5, 12.5, 12.5, 12.5) \\
\sum c_i &= 100, \quad \frac{1}{2}\sum c_i = 50 \\
E &= 20 \leq 50 \quad \text{(Use Case 1: CEA)}\\
\sum h_i &= 50 > 20 \\
4\lambda &= 20 \implies \lambda = 5 \\
\text{Allocation} &= \boxed{(5, 5, 5, 5)}
\end{align*}

\textbf{Case 2:} $c = (30, 30, 20, 20)$, $E = 20$
\begin{align*}
h &= (15, 15, 10, 10) \\
\sum c_i &= 100, \quad \frac{1}{2}\sum c_i = 50 \\
E &= 20 \leq 50 \quad \text{(Use Case 1: CEA)}\\
\sum h_i &= 50 > 20 \\
4\lambda &= 20 \implies \lambda = 5 \\
\text{Allocation} &= \boxed{(5, 5, 5, 5)}
\end{align*}

\textbf{Case 3:} $c = (40, 20, 20, 20)$, $E = 20$
\begin{align*}
h &= (20, 10, 10, 10) \\
\sum c_i &= 100, \quad \frac{1}{2}\sum c_i = 50 \\
E &= 20 \leq 50 \quad \text{(Use Case 1: CEA)}\\
\sum h_i &= 50 > 20 \\
4\lambda &= 20 \implies \lambda = 5 \\
\text{Allocation} &= \boxed{(5, 5, 5, 5)}
\end{align*}

\textbf{Case 4:} $c = (50, 20, 15, 15)$, $E = 20$
\begin{align*}
h &= (25, 10, 7.5, 7.5) \\
\sum c_i &= 100, \quad \frac{1}{2}\sum c_i = 50 \\
E &= 20 \leq 50 \quad \text{(Use Case 1: CEA)}\\
\sum h_i &= 50 > 20 \\
4\lambda &= 20 \implies \lambda = 5 \\
\text{Allocation} &= \boxed{(5, 5, 5, 5)}
\end{align*}

\textbf{Case 5:} $c = (10, 30, 30, 30)$, $E = 20$
\begin{align*}
h &= (5, 15, 15, 15) \\
\sum c_i &= 100, \quad \frac{1}{2}\sum c_i = 50 \\
E &= 20 \leq 50 \quad \text{(Use Case 1: CEA)}\\
\sum h_i &= 50 > 20 \\
4\lambda &= 20 \implies \lambda = 5 \\
\text{Allocation} &= \boxed{(5, 5, 5, 5)}
\end{align*}

\textbf{Case 6:} $c = (15, 35, 25, 25)$, $E = 20$
\begin{align*}
h &= (7.5, 17.5, 12.5, 12.5) \\
\sum c_i &= 100, \quad \frac{1}{2}\sum c_i = 50 \\
E &= 20 \leq 50 \quad \text{(Use Case 1: CEA)}\\
\sum h_i &= 50 > 20 \\
4\lambda &= 20 \implies \lambda = 5 \\
\text{Allocation} &= \boxed{(5, 5, 5, 5)}
\end{align*}

\textbf{Case 7:} $c = (60, 15, 15, 10)$, $E = 20$
\begin{align*}
h &= (30, 7.5, 7.5, 5) \\
\sum c_i &= 100, \quad \frac{1}{2}\sum c_i = 50 \\
E &= 20 \leq 50 \quad \text{(Use Case 1: CEA)}\\
\sum h_i &= 50 > 20 \\
4\lambda &= 20 \implies \lambda = 5 \\
\text{Allocation} &= \boxed{(5, 5, 5, 5)}
\end{align*}

\textbf{Case 8:} $c = (70, 10, 10, 10)$, $E = 20$
\begin{align*}
h &= (35, 5, 5, 5) \\
\sum c_i &= 100, \quad \frac{1}{2}\sum c_i = 50 \\
E &= 20 \leq 50 \quad \text{(Use Case 1: CEA)}\\
\sum h_i &= 50 > 20 \\
4\lambda &= 20 \implies \lambda = 5 \\
\text{Allocation} &= \boxed{(5, 5, 5, 5)}
\end{align*}

\section*{E = 25}

\textbf{Case 1:} $c = (25, 25, 25, 25)$, $E = 25$
\begin{align*}
h &= (12.5, 12.5, 12.5, 12.5) \\
\sum c_i &= 100, \quad \frac{1}{2}\sum c_i = 50 \\
E &= 25 \leq 50 \quad \text{(Use Case 1: CEA)}\\
\sum h_i &= 50 > 25 \\
4\lambda &= 25 \implies \lambda = 6.25 \\
\text{Allocation} &= \boxed{(6.25, 6.25, 6.25, 6.25)}
\end{align*}

\textbf{Case 2:} $c = (30, 30, 20, 20)$, $E = 25$
\begin{align*}
h &= (15, 15, 10, 10) \\
\sum c_i &= 100, \quad \frac{1}{2}\sum c_i = 50 \\
E &= 25 \leq 50 \quad \text{(Use Case 1: CEA)}\\
\sum h_i &= 50 > 25 \\
4\lambda &= 25 \implies \lambda = 6.25 \\
\text{Allocation} &= \boxed{(6.25, 6.25, 6.25, 6.25)}
\end{align*}

\textbf{Case 3:} $c = (40, 20, 20, 20)$, $E = 25$
\begin{align*}
h &= (20, 10, 10, 10) \\
\sum c_i &= 100, \quad \frac{1}{2}\sum c_i = 50 \\
E &= 25 \leq 50 \quad \text{(Use Case 1: CEA)}\\
\sum h_i &= 50 > 25 \\
4\lambda &= 25 \implies \lambda = 6.25 \\
\text{Allocation} &= \boxed{(6.25, 6.25, 6.25, 6.25)}
\end{align*}

\textbf{Case 4:} $c = (50, 20, 15, 15)$, $E = 25$
\begin{align*}
h &= (25, 10, 7.5, 7.5) \\
\sum c_i &= 100, \quad \frac{1}{2}\sum c_i = 50 \\
E &= 25 \leq 50 \quad \text{(Use Case 1: CEA)}\\
\sum h_i &= 50 > 25 \\
4\lambda &= 25 \implies \lambda = 6.25 \\
\text{Allocation} &= \boxed{(6.25, 6.25, 6.25, 6.25)}
\end{align*}

\textbf{Case 5:} $c = (10, 30, 30, 30)$, $E = 25$
\begin{align*}
h &= (5, 15, 15, 15) \\
\sum c_i &= 100, \quad \frac{1}{2}\sum c_i = 50 \\
E &= 25 \leq 50 \quad \text{(Use Case 1: CEA)}\\
\sum h_i &= 50 > 25 \\
5 + 3\lambda &= 25 \implies \lambda = \frac{20}{3} \\
\text{Allocation} &= \boxed{\left(5, \frac{20}{3}, \frac{20}{3}, \frac{20}{3}\right)}
\end{align*}

\textbf{Case 6:} $c = (60, 15, 15, 10)$, $E = 25$
\begin{align*}
h &= (30, 7.5, 7.5, 5) \\
\sum c_i &= 100, \quad \frac{1}{2}\sum c_i = 50 \\
E &= 25 \leq 50 \quad \text{(Use Case 1: CEA)}\\
\sum h_i &= 50 > 25 \\
3\lambda + 5 &= 25 \implies \lambda = \frac{20}{3} \\
\text{Allocation} &= \boxed{\left(\frac{20}{3}, \frac{20}{3}, \frac{20}{3}, 5\right)}
\end{align*}

\section*{E = 30}

\textbf{Case 1:} $c = (25, 25, 25, 25)$, $E = 30$
\begin{align*}
h &= (12.5, 12.5, 12.5, 12.5) \\
\sum c_i &= 100, \quad \frac{1}{2}\sum c_i = 50 \\
E &= 30 \leq 50 \quad \text{(Use Case 1: CEA)}\\
\sum h_i &= 50 > 30 \\
4\lambda &= 30 \implies \lambda = 7.5 \\
\text{Allocation} &= \boxed{(7.5, 7.5, 7.5, 7.5)}
\end{align*}

\textbf{Case 2:} $c = (30, 30, 20, 20)$, $E = 30$
\begin{align*}
h &= (15, 15, 10, 10) \\
\sum c_i &= 100, \quad \frac{1}{2}\sum c_i = 50 \\
E &= 30 \leq 50 \quad \text{(Use Case 1: CEA)}\\
\sum h_i &= 50 > 30 \\
4\lambda &= 30 \implies \lambda = 7.5 \\
\text{Allocation} &= \boxed{(7.5, 7.5, 7.5, 7.5)}
\end{align*}

\textbf{Case 3:} $c = (40, 20, 20, 20)$, $E = 30$
\begin{align*}
h &= (20, 10, 10, 10) \\
\sum c_i &= 100, \quad \frac{1}{2}\sum c_i = 50 \\
E &= 30 \leq 50 \quad \text{(Use Case 1: CEA)}\\
\sum h_i &= 50 > 30 \\
4\lambda &= 30 \implies \lambda = 7.5 \\
\text{Allocation} &= \boxed{(7.5, 7.5, 7.5, 7.5)}
\end{align*}

\textbf{Case 4:} $c = (50, 20, 15, 15)$, $E = 30$
\begin{align*}
h &= (25, 10, 7.5, 7.5) \\
\sum c_i &= 100, \quad \frac{1}{2}\sum c_i = 50 \\
E &= 30 \leq 50 \quad \text{(Use Case 1: CEA)}\\
\sum h_i &= 50 > 30 \\
4\lambda &= 30 \implies \lambda = 7.5 \\
\text{Allocation} &= \boxed{(7.5, 7.5, 7.5, 7.5)}
\end{align*}

\textbf{Case 5:} $c = (10, 30, 30, 30)$, $E = 30$
\begin{align*}
h &= (5, 15, 15, 15) \\
\sum c_i &= 100, \quad \frac{1}{2}\sum c_i = 50 \\
E &= 30 \leq 50 \quad \text{(Use Case 1: CEA)}\\
\sum h_i &= 50 > 30 \\
5 + 3\lambda &= 30 \implies \lambda = \frac{25}{3} \\
\text{Allocation} &= \boxed{\left(5, \frac{25}{3}, \frac{25}{3}, \frac{25}{3}\right)}
\end{align*}

\textbf{Case 6:} $c = (35, 35, 15, 15)$, $E = 30$
\begin{align*}
h &= (17.5, 17.5, 7.5, 7.5) \\
\sum c_i &= 100, \quad \frac{1}{2}\sum c_i = 50 \\
E &= 30 \leq 50 \quad \text{(Use Case 1: CEA)}\\
\sum h_i &= 50 > 30 \\
4\lambda &= 30 \implies \lambda = 7.5 \\
\text{Allocation} &= \boxed{(7.5, 7.5, 7.5, 7.5)}
\end{align*}

\section*{Extreme Cases}

\textbf{Case 1:} $c = (90, 5, 3, 2)$, $E = 20$
\begin{align*}
h &= (45, 2.5, 1.5, 1) \\
\sum c_i &= 100, \quad \frac{1}{2}\sum c_i = 50 \\
E &= 20 \leq 50 \quad \text{(Use Case 1: CEA)}\\
\sum h_i &= 50 > 20 \\
\lambda + 2.5 + 1.5 + 1 &= 20 \implies \lambda = 15 \\
\text{Allocation} &= \boxed{(15, 2.5, 1.5, 1)}
\end{align*}

\textbf{Case 2:} $c = (1, 33, 33, 33)$, $E = 20$
\begin{align*}
h &= (0.5, 16.5, 16.5, 16.5) \\
\sum c_i &= 100, \quad \frac{1}{2}\sum c_i = 50 \\
E &= 20 \leq 50 \quad \text{(Use Case 1: CEA)}\\
\sum h_i &= 50 > 20 \\
0.5 + 3\lambda &= 20 \implies \lambda = 6.5 \\
\text{Allocation} &= \boxed{(0.5, 6.5, 6.5, 6.5)}
\end{align*}

\textbf{Case 3:} $c = (70, 10, 10, 10)$, $E = 25$
\begin{align*}
h &= (35, 5, 5, 5) \\
\sum c_i &= 100, \quad \frac{1}{2}\sum c_i = 50 \\
E &= 25 \leq 50 \quad \text{(Use Case 1: CEA)}\\
\sum h_i &= 50 > 25 \\
\lambda + 5 + 5 + 5 &= 25 \implies \lambda = 10 \\
\text{Allocation} &= \boxed{(10, 5, 5, 5)}
\end{align*}

\textbf{Case 4:} $c = (80, 10, 5, 5)$, $E = 30$
\begin{align*}
h &= (40, 5, 2.5, 2.5) \\
\sum c_i &= 100, \quad \frac{1}{2}\sum c_i = 50 \\
E &= 30 \leq 50 \quad \text{(Use Case 1: CEA)}\\
\sum h_i &= 50 > 30 \\
\lambda + 5 + 2.5 + 2.5 &= 30 \implies \lambda = 20 \\
\text{Allocation} &= \boxed{(20, 5, 2.5, 2.5)}
\end{align*}

\section*{Note}

In all test cases, $\sum c_i = 100$ and $E \in \{20, 25, 30\}$, so $E \leq 50 = \frac{1}{2}\sum c_i$. Therefore, all cases use Case 1 (CEA on half-claims), which gives the same results as CG-CEA.

\end{document}
